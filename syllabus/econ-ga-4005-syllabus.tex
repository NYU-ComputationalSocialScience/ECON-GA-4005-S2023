%\documentclass[11pt]{article}

%%%%%%%%%%%%%%%%%%%%%% Stock Preamble %%%%%%%%%%%%%%%%%%%%%%%%%%%%%%%%%%
\documentclass[12pt,pdftex,twoside,letterpaper]{exam}

\usepackage{amsmath,amssymb, amsthm}
\usepackage{graphicx}
\usepackage{color}
\usepackage{comment}
\usepackage{layout}
\usepackage{booktabs}
\usepackage[flushleft]{threeparttable}
\usepackage{caption}
\usepackage{setspace}
\usepackage{float}
\usepackage{needspace}
\usepackage[colorlinks=true,linkcolor=blue,citecolor=black,urlcolor=blue,bookmarks=false,pdfstartview={FitV}]{hyperref}

%%%%%%%%%%%%%%%%%%%%%%Margins%%%%%%%%%%%%%%%%%%%%%%%%%%%%%%%%%%%%%%%%%%%%%%%
\usepackage[margin=1.0in]{geometry}
\setlength{\parindent}{0in}
\setlength{\parskip}{.09in}
\raggedbottom

%%%%%%%%%%%%%%%%%%%%Tighten up the lists%%%%%%%%%%%%%%%%%%%%%%%%%%%%%%%%%%
\let\OLDdescription\description
\renewcommand\description{\OLDdescription\setlength{\itemsep}{-2mm}}

%%%%%%%%%%%%%%%%%%%%%%%%%Exam class formating%%%%%%%%%%%%%%%%%%%%%%%%%%%%%%%%%%%%%%%%
\renewcommand{\partlabel}{\thepartno.}
\renewcommand{\questionshook}{\setlength{\itemsep}{0.2in}}
\renewcommand{\partshook}{\setlength{\leftmargin}{0.2in}}
\renewcommand\familydefault{\sfdefault}

%%%%%%%%%%%%%%%%%%%%%%%%%%%%%%%%%Let Backus fight the good fight, I'm going with LN%%%%%%%%%%%%%%%%%%%
\renewcommand{\log}{\ln}

%%%%%%%%%%%%%%%%%%%%%%Headers and Footers%%%%%%%%%%%%%%%%%%%%%%%%%%%%%%%%%%
\pagestyle{headandfoot}

\runningheadrule
\firstpageheadrule
\runningheader{ML}{}{Syllabus: ECON-GA 4005}
\runningfooter{}{}{}

\begin{document}

  \centerline{\Large\bf Syllabus: Machine Learning and Algorithms for CSS}
  \vspace{3mm}
  \centerline{\large\bf ECON-GA 4005 $|$ Spring 2023}
  \vspace{3mm}
  \centerline{\bf Revised: \today}

  \bigskip

  When applied correctly, "machine learning" tools allow individuals to approximate complicated
  outcomes in the real world. However, when applied carelessly, these tools generate misleading
  findings. This course covers supervised learning (both regression and classification),
  reinforcement learning, and model selection via validation procedures. This course prepares
  students to apply classical and cutting-edge machine learning techniques to problems in the
  social sciences.

  The course presents a principled approach that adheres to best practices and encourages
  understanding and transparency.

  This class will use Python, a popular high-level computer language, that is being used widely
  across many fields. ``High-level'' means it's less painful than most (the hard work is done by
  the language), but it's a serious language with extensive capabilities.

  \subsubsection*{About the instructors}

    This course is co-taught by three instructors: Chase Coleman (cgc332@nyu.edu), Spencer Lyon
    (sgl290@nyu.edu), and Thomas Sargent (ts43@nyu.edu).

    Office hours: By appointment

  \subsubsection*{Where and When}

    Meeting times: Thursday 6:00 pm - 8:50 pm (Eastern Standard Time)

    Meeting place: \href{https://app.gather.town/app/70NVJBXquyIwqiOL/NYU%20CSS}{Gather} (password: \texttt{nyu compsosci} -- best to copy/paste)

  \section*{Requirements}

    The main prerequisites for this course are a solid mastery of the concepts covered in the
    Mathematical Foundations for Computational Social Science (ECON-GA-4002) and the Data Skills for Computational (ECON-GA-4003)
    Social Science courses. See the corresponding course syllabi for a list of topics covered in
    these courses. If you have not taken the courses, we will expect that you either have commensurate experience or are willing to work hard to gain it early in the semester.

    A willingness to consult Wikipedia for math and statistics concepts will also be helpful for
    students.

  \section*{Getting help}

    This course is meant to be collaborative and to have a strong support system to help you when
    you run into problems. Please use the discussion board or other forms of communication to
    communicate with your classmates and ask questions/help each other.

    The bottom line:  {\bf If you're stuck, ask for help\/}.

    Really.  Don't be a hero, ask for help.

  \section*{Course materials and assignments}

    All course materials will be posted on \href{https://github.com/NYU-ComputationalSocialScience/ECON-GA-4005-S2023}{Github}.

    Assignments, including your exam, will be assigned through Brightspace

  \section*{Deliverables and grades}

    Graded work includes:

    \begin{itemize}
      \item {\bf Code Practice (30\%)} There will be several homework assignments throughout the
            semester. We find that people who finish these assignments tend to keep up
            with the material better and these are easy points to get in terms of grades.
      \item {\bf Exam (20\%)} There will be one exam. The exam will be take home and you may use
            your notes and the internet. The only exception is that we'd ask that you don't ask
            the exam questions in online forums and that you don't speak with your classmates about
            the exam.
      \item {\bf Project (40\%)} The most important deliverable, both in terms of your grade and
            what you will learn, is a project that applies the tools from class to some question
            in the social sciences.
      \item {\bf Participation and Quizzes (10\%)} We expect students to participate in class. If you have
            questions, don't be shy about asking them. The discussion forums are a great place to
            participate in class as well. We will also have quizzes from time to time. These will not be announced beforehand and will only cover material from previous class sessions. This is a mechanism designed to help you stay up to speed with the topics and content.
    \end{itemize}


    {\bf Due dates} will be posted on the Brightspace.

    {\bf Dates are not negotiable. Anything handed in late will get a grade of zero.}

    {\bf All your work should be clean and professional.} We expect your math to be written in
    LaTeX for this course and expect your code to conform to good code habits.

    {\bf Final grades\/} will be computed from

    \begin{center}
      \begin{tabular}{ll}
        Participation / Quizzes & 10\% \\
        Code practice & 30\% \\
        Exam \#1        & 20\% \\
        Project        & 40\% \\
      \end{tabular}
    \end{center}

    Final grades are not subject to any fixed distribution or curve. The number of A grades, for
    example, will depend only on your performance in the course.

\section*{Recommended work habits}

  Python is not something you can learn from reading a book and attending lectures. You need to
  {\bf write programs}... The more the better. Think about how you'd learn to play basketball
  or soccer; reading and listening to lectures aren't enough, you need to do it. We'll do a lot of
  programming in class, but it's {essential} that you follow up outside of class.

\section*{Pacing}

  The course is designed to be cover material at whatever pace the class is capable of. The topics
  should take roughly one week each, but we can scale that up or down as needed. If you're an expert,
  don't worry, we'll cover a lot of material either way.

\section*{Other questions}

  We encourage students who have questions to typically post their questions on the Brightspace
  site so that answers can be referenced by other students. If no answer is provided in a reasonable
  amount of time (i.e. wait at least 24 hours), you may email us to remind us of the question. If you
  have a question about a matter that should be kept private, please don't hesitate to reach out
  directly by email.


\section*{Policies}

\begin{itemize}
  \item \textbf{General Behavior.} The School expects that students will conduct themselves with
        respect and professionalism toward faculty, students, and others present in class and will
        follow the rules laid down by the instructor for classroom behavior.  Students who fail to
        do so may be asked to leave the classroom.

  \item \textbf{Collaboration on Graded Assignments.} You may discuss assignments with anyone
        (in fact, we encourage it), but anything you submit, including your code, should be your
        own. Exams should be entirely your own work. Violation of this policy will result in a
        failing grade for the course.

  \item \textbf{Academic Integrity.}

  \begin{itemize}
    \item Integrity is critical to the learning process and to all that we do here at NYU. As
          members of our community, all students agree to abide by the NYU Student Code of Conduct,
          which includes a commitment to:
    \item Exercise integrity in all aspects of one's academic work including, but not limited to,
          the preparation and completion of exams, papers and all other course requirements by not
          engaging in any method or means that provides an unfair advantage.
    \item Clearly acknowledge the work and efforts of others when submitting written work as one's
          own. Ideas, data, direct quotations (which should be designated with quotation marks),
          paraphrasing, creative expression, or any other incorporation of the work of others
          should be fully referenced.
    \item Refrain from behaving in ways that knowingly support, assist, or in any way attempt to
          enable another person to engage in any violation of the Code of Conduct. Our support also
          includes reporting any observed violations of this Code of Conduct or other School and
          University policies that are deemed to adversely affect the NYU community.
  \end{itemize}

  The entire Student \href{https://www.nyu.edu/about/policies-guidelines-compliance/compliance/code-of-ethical-conduct.html}{Code of Conduct}
  applies to all students enrolled in NYU courses. \textbf{Any violation of the a policies
  pertaining to Academic Integrity will result in a failing grade for the course.}
\end{itemize}

\subsubsection*{Students with disabilities}

  If you have a qualified disability that requires academic accommodation during this course,
  please contact the Moses Center for Students with Disabilities (CSD, 212-998-4980) and ask them
  to send me a letter verifying your registration and outlining the accommodation they recommend.
  If you need to take an exam at the CSD, you must submit a completed Exam Accommodations Form to
  them at least one week prior to the scheduled exam time to be guaranteed accommodation.

\end{document}
